\documentclass[9pt,serif]{beamer}
\usepackage[]{tikz}
\usepackage[]{graphicx}
\usepackage[]{float} 
\usepackage[]{import}
\usetheme{Rochester}
\usepackage{graphicx}
\usepackage[]{amsthm,amssymb,amsmath}


%

\usepackage[icmcb]{mycolors} 
\usepackage{verbatim}


    \usepackage{thmtools, listings, chngcntr}
    \usepackage[most]{tcolorbox}
    \usepackage{ifthen, xifthen}
    \usepackage[framemethod=TikZ]{mdframed}

%This code is made for Itemized portions
\newenvironment{myitemize}{\begin{itemize}}{\end{itemize}}
    \tcolorboxenvironment{myitemize}{blanker, before skip=6pt,after skip=6pt,
borderline west={2mm}{0pt}{red}} 

    \declaretheoremstyle[ headfont=\bfseries\color{black}, notebraces={[}{]},
    notefont=\normalfont\sffamily, headpunct={. }, spaceabove=10pt,
    spacebelow=10pt, qed=\qedsymbol, bodyfont=\color{black}, ]{solutionstyle}

\tcolorboxenvironment{solution.}{ coltext=solC!90!pageC, tile, breakable,
colback=pageC, top=0pt, left=.7em, right=.7em, before skip=0pt,after
skip=35pt,borderline west={.1em}{0pt}{solC}, }
\declaretheorem[style=solutionstyle]{solution.} \newtheorem*{solution.*}{Sl}


        \declaretheoremstyle[ headfont=\bfseries\color{Olive Green},
        headpunct={.}, postheadspace={7pt}, spaceabove=12pt,
        bodyfont=\normalfont, spacebelow=12pt, notebraces={(}{)},
        notefont=\color{probC}, ]{ideastyle}


\tcolorboxenvironment{idea}{ tile, breakable,colback=defBgC, colframe=probBgC,
coltext=fgC, left=1em,before skip=12pt,after skip=30pt, borderline
west={2mm}{0pt}{gray}, } \declaretheorem[style=ideastyle]{idea}
\newtheorem*{idea*}{id}


        \declaretheoremstyle[ headfont=\bfseries\color{probC}, headpunct={.},
        postheadspace={7pt}, spaceabove=12pt, bodyfont=\normalfont,
        spacebelow=10pt, notebraces={(}{)}, notefont=\color{probC},
        ]{problemstyle}


\tcolorboxenvironment{problem.}{ tile, breakable,colback=probBgC,
colframe=probBgC, coltext=fgC, left=1em,before skip=12pt,after skip=30pt, }
\declaretheorem[style=problemstyle]{problem.} \newtheorem*{problem.*}{Pr}

         \newcommand{\solu}[1]{ \begin{solution.} #1 \end{solution.} }
         \newcommand{\prob}[1]{ \begin{problem.} #1 \end{problem.}}

        \newcommand{\id}[1]{ \begin{idea} #1 \end{idea} }
%_{_{_{_{_}}}} Macros are here 
\newcommand{\fig}[3]{ \begin{figure} \centering \includegraphics[width=
#1\textwidth]{#2} \caption{#3}  \end{figure}} 
%
\newcommand{\draw}[3]{ \begin{figure}[hbt!] \centering
 \fontsize{35pt}{20pt}\selectfont \resizebox{#1 \textwidth}{!}{
\import{}{#2.pdf_tex}}\caption{#3} \label{#3} \end{figure} }
%_{_{_{_{}}}}}

\newcommand{\sides}[2]{ \begin{minipage}{0.5\textwidth} #1
    \end{minipage}\hfill%
\begin{minipage}{0.5\textwidth} #2 \end{minipage} }
\usepackage[]{xcolor}
 \usepackage[]{pdfpages}
 \usepackage[]{transparent}
 \usepackage[]{xifthen}
\newcommand{\incfig}[1]{ 
\fontsize{ 35pt}{ 20pt}\selectfont \resizebox{0.6\textwidth}{!}{ 
\import{./figures/}{#1.pdf_tex} } 
} 

\title{Day 1} \author{Ahmed Saad Sabit} \date{\today} 



\begin{document}

% this is the title page.
\begin{frame}{Intro} \titlepage  \end{frame}


\begin{frame}
    {Problem 1: Find Equivalent Spring 
    Constant}
    \prob{We have two spring of $k_1$ and $k_2$ constant. IF we join them in series, find the equivalent $k$ for the system.}
\end{frame}


\begin{frame}
    {Spring}
    \sides{For a spring, if we pull it with $F$, 
        \[ F = k \Delta x \]
    }{ \draw{0.9}{prob1-1}{}} 
\end{frame}

\begin{frame}
    \sides{How spring work, a spring pulls the both sides of it's ends with the same force.}{\draw{0.9}{prob1-2}{}}
\end{frame}


\begin{frame}
    {The visual of the problem}
    \draw{0.6}{prob1-3}{}
\end{frame}


\begin{frame}
    {Force balance conditions, system is going to be at rest}
    \draw{0.7}{prob1-4}{} 
\end{frame}
    

\begin{frame}
    \begin{itemize}
        \item The two spring together is as a single spring.
        \item So, the ends of the whole big spring will pull with $ F$ and $ F$ at both sides.
        \item $k_1$ spring will pull it's two ends with $F_1$ and $k_2$ will pull with $F_2$.
    \end{itemize}
    
\end{frame}


\begin{frame}
    {Solving}

        Note that, two springs expand $\Delta x_1$ and $\Delta x_2$. And total expanding, 
        \[ \Delta x = \Delta x_1 + \Delta x_2\]
       And this is clear, 
       \[ F_1 = k_1 \Delta x_1  
       \quad F_2 = k_2 \Delta x_2 \]
    \sides{

       Now, the whole spring system is at rest, so force is balanced at every points in the spring. That's why, All the pair forces cancel out.
        \[ F = F_1 \quad F_1 = F_2 \quad F = F_2 \]
        This means, 
        $ F_1 = F_2 $
        Thus, \[ k_1 \Delta x_{1} = k_2 \Delta x_2 \]
        Solves to, \[ \Delta x_{2} = \frac{k_1}{k_2} \Delta x_1 \]
    }{\draw{1}{prob1-4}{}}
\end{frame}


\begin{frame}
    {Solution to $1$ }
    Now, we applied a total force $F$, then total length expansion $\Delta x$ and thus, equivalent $k$ is, 
    \[ \frac{F}{\Delta x} = k \]
    Because $F = F_1 = F_2$, 
    \[ \frac{k_1 \Delta x_1}{\Delta x_1 + \Delta x_2} = k \]
   \[ \frac{k_1 \Delta x_1}{ \Delta x_1 + \frac{k_1}{k_2} \Delta x_1 } = k \]
   Solving this, we get, 
  \[ k = \frac{k_1 k_2 }{k_1 + k_2} \]
   Answer, 
   \[ \boxed{ \frac{1}{k} = \frac{1}{k_1} + \frac{1}{k_2} = \sum_{i} \frac{1}{k_i}} \]
   
   
\end{frame}


\begin{frame}
    {Center of Mass}
    Let us have two masses $m_1$ and $m_2$. They are located along $x_1 $ and $x_2$. So, their center of mass will be at, 
    
    \[ x_c = \frac{m_1x_1 + m_2 x_2 }{m_1 + m_2} \]
    If you take a derivative of  this with respect to time, 

    \[ \frac{\mathrm{d} }{\mathrm{d} t} x_c = \frac{\mathrm{d} }{\mathrm{d} t} \left(\frac{m_1x_1 + m_2 x_2 }{m_1 + m_2}\right) \] 
    We find, 
    \[ v_c = \frac{m_1 v_1 + m_2 v_2}{ m_1 + m_2} \]
    That is, 
    \[ \left( m_1 + m_2 \right) v_c = m_1 v_1 + m_2 v_2 \]
    And is consistent with the momentum idea we discussed in day 2. 
\end{frame}

\begin{frame}
    {Center of Mass along one axis}
    \draw{0.7}{com}{}
\end{frame}

\begin{frame}
    {Vectorally}
    \[ \vec r = \frac{m_1 \vec r_1 + m_2 \vec r_2 }{m_1 + m_2} \]
\pause
That means, 
\[ x_c =   \frac{m_1x_1 + m_2 x_2 }{m_1 + m_2} \]
\[ y_c =   \frac{m_1y_1 + m_2 y_2 }{m_1 + m_2} \]
\[ z_c =  \frac{m_1z_1 + m_2 z_2 }{m_1 + m_2}  \]
\pause
Now if you take small portions of masses, and want to do it continuously,
\[ x_c = \frac{\int x \ \mathrm{d} m}{M} \]

\end{frame}


\begin{frame}
    {Collition and Center of Mass}
    If there is no external force working, then the center of mass will not move from it's place, or it will keep moving in a constant speed.  
    \sides{
    \prob{Suppose there is a mass $M$ that is resting in it's position, and another mass $m$ is moving towards it at $v_m$ speed. Now if they collide, find each of their speed after collision. All motion takes place in $x$ axis. There is no loss of energy.}}{\draw{1}{prob2-1}{}} 
\end{frame}

\begin{frame}
    {Momentum Conservation solution}
    \[ mv = Mv_M + mv_m \]
   \[ \frac{1}{2} m v^2 = \frac{1}{2} M v_M^2 + \frac{1}{2} mv_m \]
   Number of unknown, $v_M , v_m$, \\
   Number of equation, $2$.
\end{frame}

\begin{frame}
    {Postpone this problem for now.}
    Think about another problem where the center of mass is not moving. Thus, we can have this diagram.
    \draw{0.6}{prob2-2}{}
    Here, the center of mass will stay where it is, and unless no external force acts there will not be any movement. \textbf{Now how would the motion look after the masses have collided?} The center of mass won't move anyway. 
\end{frame}


\begin{frame}
    \draw{0.4}{prob2-2}{}
    Because center of mass not moving, 
    \[ 0 = M v_M - mv_m \]
    After collision, the masses should move away. Say at $v_M', v_m'$ speed.
    \[ 0 = -M v_M' + mv_m' \]
   Energy is conserved too. What if $v_M = - v_M'$? and $v_m = - v_m'$? This is exactly the opposite motion and the center of mass is kept in it's position too.  
\end{frame}

\begin{frame}
    {Reasons}
    \begin{itemize}
        \item If energy is conserved, then the where will the speed of the particles go? 
        \item If one particle get extra speed from other particle, then the center of mass will not be in it's position as it is.
    \end{itemize}
\end{frame}

\begin{frame}
    {After collision, motion is just reveresed in respect to the Center of Mass}
    \draw{0.4}{prob2-2}{}
    \draw{0.4}{prob2-3}{In the frame of Center of Mass in rest, the speed is just reversed after collision, if energy is conserved.}
\end{frame}

\begin{frame}
    \id{The speed of particles after collision is just reversed in the Center of Mass frame, if the collision doesn't lose energy.}
    Not losing energy means \textbf{Elastic Collsion}. And Losing is \textbf{Inelastic}. 
\end{frame}

\begin{frame}
    {Back to the problem}
    \prob{Suppose there is a mass $M$ that is resting in it's position, and another mass $m$ is moving towards it at $v_m$ speed. Now if they collide, find each of their speed after collision. All motion takes place in $x$ axis. There is no loss of energy.\pause}
    {\begin{small}Let us move in the frame of center of mass, speed of center of mass is, 
        \[ v_c = \frac{m v_m}{M + m} \]
        Relative speed of anything, 
        \[ \text{Relative Speed} =\text{ Actual Speed} -\text{ Frame Speed} \]
        \[ u_m = v_m - v_c = v_m - \frac{mv_m}{M + m} = \frac{M v_m}{M + m} \]
       \[ u_M = v_M - v_c = 0 - \frac{mv_m}{M + m} =  - \frac{mv_m}{M + m}  \]
        
\end{small}} 
\end{frame}

\begin{frame}
    {After the collision}
    \[ v_m \to -v_m \]
    \[ v_M \to -v_M \]
    Thus, after collision,
    \[ u_m' = - \frac{M v_m}{M + m}  \]
    \[ u_M ' = \frac{mv_m}{M+m} \]
    Now, we can use the equation for relative speed, 
    
        \[ \text{Relative Speed} =\text{ Actual Speed} -\text{ Frame Speed} \]
        \[ \text{Relative Speed} + \text{Frame Speed} = \text{Actual Speed} \]
    
        \[ v_m'  = -\frac{Mv_m}{M+m} +   \frac{mv_m}{M+m } = \frac{\left( m - M \right) v_m}{M + m}       \]
        \[ v_M' =  \frac{mv_m}{M+m}  +  \frac{mv_m}{M+m} =  \frac{2mv_m}{M+m}   \]
        
\end{frame}
\begin{frame}
    {Solution to 2nd problem}
    \begin{align*}
        v_m' &= \frac{\left( m - M \right) v_m}{M + m} \\
        v_M' &=  \frac{2mv_m}{M+m} 
    \end{align*}
    
\end{frame}


\begin{frame}
    \textbf{Theorem :   } In a $1-D$ Elastic collision, the relative velocity of two particles after a collsion is the negative of the relative velocity before the collision, in the Center of Mass frame. \\
    \textbf{Proof :} This is in Morin $4.7$, I will upload the text book in the Classroom, you are requested to check that. Basic idea is using energy and momentum conservation together.
\end{frame}









\begin{frame}
    \[ \frac{\mathrm{d} f}{\mathrm{d} x} \]
     \[ \frac{\mathrm{d} }{\mathrm{d} } \]
     \[ \begin{pmatrix} 2 & 2 & 2 \\ 
     2 & 4 & 3 \\
 2 & 4 & 2\end{pmatrix}  \]
     \[ \frac{1}{2} \sqrt{2x^2+2^3}  \]
      
\end{frame}



\end{document}
